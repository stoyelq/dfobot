Obtaining age estimates from hard structures such as fish otoliths is a key component of stock assessments methods that examine population dynamics using age-structured models. Traditionally, age estimates are obtained from a representative set of individuals in order to derive an age-length key that can then be used to estimate the age composition of a population based on the distribution of lengths.

Age estimates are obtained from otoliths through the examination of regular concentric patterns associated with the alternation of opaque and hyaline zones. In temperate systems, these patterns often correspond to yearly changes in environmental conditions and can be used to estimate age.

The number of age estimates required to obtain an unbiased age-length key is often large and it is not uncommon for thousands of fish age estimates to be determined by fisheries technicians in support of age-based assessments. 

We apply a neural network approach to build a deep learning system that uses otolith images to estimate ages. The model is developed using otoliths from seven marine fish species from the southern Gulf of St. Lawrence. 


\cite{moen-2018}, \cite{polacek-2023}, \cite{politikos-2021}, \cite{sigurdardottir-2023}
 
